\documentclass[a5paper,12pt,twoside]{memoir}
\usepackage{fontspec}
\defaultfontfeatures{Ligatures=TeX}
%\setsansfont{Quercus}
\setmainfont{Calluna}
\usepackage{gregoriotex}
\usepackage[cm]{fullpage}
\begin{document}


\setgrefactor{18}

\centerline{\Large Sing, my tongue, the Saviour's glory}

\smallskip

\centerline{\large Pange Lingua by St Thomas Aquinas}

\smallskip


\centerline{\large translated by Fr Edward Caswall}

\includescore{singmytong.tex}

\bigskip

Written by St. Thomas Aquinas (1225--1274) for the Solemnity of Corpus Christi, this hymn is considered the most beautiful of Aquinas' hymns and one of the great seven hymns of the Church. 

The rhythm of the Pange Lingua is said to have come down from a marching song of Caesar's Legions: Ecce, Caesar nunc triumphat qui subegit Gallias. 

Besides the Solemnity of Corpus Christi, this hymn is also used on Holy Thursday. 

The last two stanzas make up the Tantum Ergo (Down in Adoration Falling) that is used at Benediction of the Blessed Sacrament.


There is another Latin hymn starting \emph{Pange lingua gloriosi} written much earlier by Fortunatus.  That is used at Good Friday and commemorates the Passion of Our Lord.

\end{document}


